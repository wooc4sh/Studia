\documentclass{article}

\usepackage{import}
\usepackage{transparent}
\usepackage{xcolor}


%Commands definitions
\newcommand{\setbackgroundcolour}{\pagecolor[rgb]{0.19,0.19,0.19}}  
\newcommand{\settextcolour}{\color[rgb]{0.87,0.87,0.87}}    
\newcommand{\invertbackgroundtext}{\setbackgroundcolour\settextcolour}

%Command execution. 
%If this line is commented, then the appearance remains as usual.
\invertbackgroundtext

%na gorze beda cale komendy usepackage etc

\author{Lukasz Kopyto}
\title{Notatki z ksiazki Algorytmy + struktury danych = programy}

\begin{document}  %begin calego dokumentu
\maketitle     %dowiedz sie o co kaman z tym tytulem, czemu tu jest make

\section{Rekurencja}
\subsection{Algorytmy z nawrotami}
\title{Droga skoczka szachowego}


Mamy szachownice $n \times n$ o $n^{2}$ polach. Na poczatku szkoczek jest umieszczany na polu $y_{0}$ i $x_{0}$. Problem polega na znalezieniu drogi o dlugosci $n^{2}-1$ ruchow, ze kazde pole bedzie dokladnie raz odwiedzone.

Pierwsze podejscie:

mat



\end{document} %end calego dokumentu
