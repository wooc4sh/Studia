\documentclass{article}

\usepackage{import} %to nie wiem po co
\usepackage{transparent} %nie pamietam, chyba do odwracania kolorow
\usepackage{xcolor} %nie pamietam, chyba do odwracania kolorow
\usepackage{tcolorbox} %wazne ramki kolorowe
\usepackage{amsmath} %wazne matematyka
\usepackage{upgreek} %nie pamietam, chyba duze litery greckie
\usepackage{enumitem} %nie pamietam, chyba to wymieniania
\usepackage{amssymb} %wazne matematyka
\usepackage[margin=0.25in]{geometry} %wazne marginesy



%Commands definitions
\newcommand{\setbackgroundcolour}{\pagecolor[rgb]{0.19,0.19,0.19}}  
\newcommand{\settextcolour}{\color[rgb]{0.87,0.87,0.87}}    
\newcommand{\invertbackgroundtext}{\setbackgroundcolour\settextcolour}

%Command execution. 
%If this line is commented, then the appearance remains as usual.
\invertbackgroundtext

%na gorze beda cale komendy usepackage etc

\author{Lukasz Kopyto}
\title{Moje notatki na bazie skryptu topology without tears}

\begin{document}  %begin calego dokumentu
\maketitle     %dowiedz sie o co kaman z tym tytulem, czemu tu jest make

\section{Przestrzenie topologiczne}
\subsection{Definicje, lematy, zadania i rozwiazania}
\begin{tcolorbox}[colback=white!90!red,colframe=black!35!red,title=1.1.1 Definicja: Topologia]

Niech $X$ bedzie niepustym zbiorem. Mowimy, ze zbior $T$ zawierajacy podzbiory $X$ jest topologia na $X$ jesli:
\begin{itemize}
\item $X$ oraz zbior pusty $\emptyset$, naleza do $T$,
\item suma skonczona badz nie zbiorow z $T$ nalezy do $T$,
inaczej $T$ jest zamkniety na sume, i ta suma moze byc nieskonczona
\item przekroj dowolnych dwoch zbiorow z $T$ nalezy do $T$, inaczej $T$ jest zamkniety na przekroj, ale przekroj skonczony
\end{itemize}

Para $(X,T)$ nazywamy \textbf{przestrzenia topologiczna.}
\end{tcolorbox}

\begin{tcolorbox}[colback=white!90!red,colframe=black!35!red,title=1.1.6 Definicja: Topologia dyskretna(discrete topology)]

Niech $X$ bedzie niepustym zbiorem oraz $T$ kolekcja wszystkich podzbiorow $X$. Wtedy mowimy, ze $T$ jest \textbf{topologia dyskretna} na zbiorze $X$. Przestrzen topologiczna $(X,T)$ jest nazywana \textbf{przestrzenia dyskretna}
\end{tcolorbox}

\begin{tcolorbox}[colback=white!90!red,colframe=black!35!red,title=1.1.7 Definicja: Topologia niedyskretne(indiscrete topology)]

Niech $X$ bedzie niepustym zbiorem oraz $T= \left\{X, \emptyset\right\}$. Wtedy mowimy, ze $T$ jest \textbf{topologia niedyskretna} oraz $(X,T)$ jest \textbf{przestrzenia niedyskretna} 

\end{tcolorbox}

\begin{tcolorbox}[colback=white!90!green,colframe=black!35!green,title=1.1.9 Lemat: Topologia dyskretna i singletony]

Jezeli $(X,T)$ jest przestrzenia topologiczna, taka ze, dla kazdego $x \in X$, singleton $x$, $\left\{x\right\}$ jest w $X$, to $X$ jest topologia dyskretna.

\end{tcolorbox}
\textbf{Dowod:}

Wystarcz sprawdzic prawdziwosc trzech warunkow, z definicji $1.1.1$.
\begin{enumerate}
\item Z definicji $T$ jest topologia, wiec zawiera $X$ oraz $\emptyset$.
\item Niech $S$ bedzie suma skonczona lub nie dowolnej liczby zbiorow z $T$. Poniewaz mozemy zapisac $S$ jako $S = \bigcup\limits_{x \in S}\left\{x\right\}$, a kazdy singleton $\left\{x\right\} \in X$, wnioskujemy stad, ze $S \in T$ 
\item Analogicznie dowodzimy, ze przekroj dowolnych dwoch zbiorow z $T$ nalezy do $T$
\end{enumerate}

\hrulefill

\textbf{Cwiczenia 1.1}

\begin{enumerate}
\item Niech $X = \left\{a,b,c,d,e,f\right\}.$ Ustal czy podane kolekcje podzbiorow $X$ sa topologia na $X$

\begin{enumerate}[label=(\alph*)]
\item $T_{1} = \left\{X, \emptyset, \left\{a\right\}, \left\{a,f\right\}, \left\{b,f\right\}, \left\{a,b,f\right\}\right\}$
\item  $T_{2} = \left\{X, \emptyset, \left\{a,b,f\right\}, \left\{a,b,d\right\}, \left\{a,b,d,f\right\}\right\}$
\item $T_{3} = \left\{X, \emptyset, \left\{f\right\}, \left\{e,f\right\}, \left\{a,f\right\} \right\}$
    
\end{enumerate}
\textbf{Odpowiedz:}
\begin{enumerate}[label=(\alph*)]
\item Jest topologia.
\item Nie jest, bo $\left\{a,b,f\right\} \cap \left\{a,b,d\right\}= \left\{a,b\right\} \notin T_{2}$ 
\item Nie jest, bo $\left\{e,f\right\} \cup \left\{a,f\right\}=\left\{a,e,f\right\} \notin T_{3}$ 
    
\end{enumerate} 
\item Niech $X = \left\{a,b,c,d,e,f\right\}.$ Wskaz i uzasadnij ktore kolekcje podzbiorow $X$ sa topologia na $X$

\begin{enumerate}[label=(\alph*)]
\item $T_{1} = \left\{X, \emptyset, \left\{c\right\}, \left\{b,d,e\right\}, \left\{b,c,d,e\right\}, \left\{b\right\}\right\}$
\item  $T_{2} = \left\{X, \emptyset,\left\{a\right\}, \left\{b,d,e\right\}, \left\{a,b,d\right\}, \left\{a,b,d,e\right\}\right\}$
\item $T_{3} = \left\{X, \emptyset,\left\{b\right\}, \left\{a,b,c\right\}, \left\{d,e,f\right\}, \left\{b,d,e,f\right\} \right\}$
 
\end{enumerate}

\textbf{Odpowiedz:}

\begin{enumerate}[label=(\alph*)]
\item Nie, bo $\left\{c\right\} \cup \left\{b\right\} = \left\{b,c\right\} \notin T_{1}$
\item Nie, bo $\left\{b,d,e\right\} \cap \left\{a,b,d\right\} = \left\{b,d\right\} \notin T_{2}$
\item Jest topologia.
\end{enumerate}

\item Niech $X = \left\{a,b,c,d,e,f\right\}$ oraz $T$ bedzie topologia dyskretna na $X$, ktore z ponizszych podpunktow sa prawdziwe?

\begin{enumerate}[label=(\alph*)]
\item $X \in T$- Prawda
\item $\left\{X\right\} \in T$- Falsz
\item $\left\{ \emptyset\right\} \in T$- Falsz
\item $\emptyset \in T$- Prawda
\item $\emptyset \in X$- Falsz 
\item $\left\{ \emptyset\right\} \in X$- Falsz
\item $\left\{a \right\} \in T$- Prawda
\item $a  \in T$- Falsz
\item $\emptyset \subseteq X$- Prawda
\item $\left\{a \right\} \in X$- Falsz
\item $\left\{\emptyset \right\} \subseteq X$- Falsz
\item $a \in X$- Prawda
\item $X \subseteq T$- Falsz
\item $\left\{a \right\} \subseteq T$- Falsz
\item $\left\{X\right\} \subseteq T$- Prawda
\item $a \subseteq T$- Falsz
\end{enumerate}

\item Niech $(X, T)$ bedzie dowolna przestrzenia topologiczna. Udowodnij, ze \textbf{przekroj skonczonej liczby(dowolnej) elementow z $T$ jest elementem $T$}

\textbf{Odpowiedz:}

Udowodnimy to przez indukcje wzgledem liczby elementow przekroju.

\textbf{Teza:} Dla przestrzeni topologicznej $(X,T)$ przekroj skonczonej liczby elementow z $T$ jest elementem $T$.

\textbf{Podstawa indukcji:} dla n = 2, mamy $\bigcap\limits_{i = 1}^{2}A_{i} \text{, gdzie } A_{i} \in T \text{, } A_{i}$ sa dowolnymi zbiorami nalezacymi do $T$. Poniewaz $T$ jest topologia na $X$ wiec z definicji, przekroj dowolnych dwoch zbiorow nalezacych do $T$, nalezy do $T$, zatem $\bigcap\limits_{i = 1}^{2}A_{i} \in T$.

\textbf{Krok indukcyjny:}
Ustalmy teraz dowolne $n \in  \mathbb{N}$ i zalozmy, ze $\bigcap\limits_{i = 1}^{n}A_{i} \in T$. Pokaze ze dla $n+1$ teza zachodzi.
Oznaczmy sobie $B = \bigcap\limits_{i = 1}^{n}A_{i}$.
Dla $n+1$ mamy:

$\bigcap\limits_{i = 1}^{n+1}A_{i} = \bigcap\limits_{i = 1}^{n}A_{i} \cap A_{n+1} = \text{ zal. ind. } = B \cap A_{n+1}$
Poniewaz $B \in T \text{ oraz } A_{n+1} \in T$ wiec z definicji topologi, $B \cap A_{n+1} \in T$  

Zatem na mocy zasady indukcji matematycznej, dla przestrzeni topologicznej $(X,T)$ przekroj skonczonej liczby elementow z $T$ jest elementem $T$.

\item Niech $\mathbb{R}$ bedzie zbiorem liczb rzeczywistych. Udowodnij ze kazdy z nastepujacych kolekcji podzbiorow $\mathbb{R}$ jest topologia.

\begin{enumerate}[label=(\alph*)]
\item $T_{1}$ zawiera $\mathbb{R}, \emptyset$ oraz kazdy przedzial $(-n, n)$, dla dowolnego $n \in {\mathbb{N}}^{+}$, gdzie $(-n, n)$ oznacza zbior $\left\{x \in \mathbb{R}: -n < x <n\right\}$
 
\item $T_{2}$ zawiera $\mathbb{R}, \emptyset$ oraz kazdy przedzial $[-n, n]$, dla dowolnego $n \in {\mathbb{N}}^{+}$, gdzie $[-n, n]$ oznacza zbior $\left\{x \in \mathbb{R}: -n \leq x \leq n\right\}$
       
\item $T_{3}$ zawiera $\mathbb{R}, \emptyset$ oraz kazdy przedzial $[n, \infty)$, dla dowolnego $n \in {\mathbb{N}^{+}}$, gdzie $[n, \infty)$ oznacza zbior $\left\{x \in \mathbb{R}: n \leq x\right\}$
    
\end{enumerate}

\textbf{Odpowiedz:}

Trzeba kazdy podpunkt sprawdzic z definicji topologii, czyli w kazdym podpunkcie sprawdzic czy (a') $X$ oraz $\emptyset$ naleza do $T$, (b') zamknietosc na sume teoriomnogosciowa skonczona lub nie i (c') zamknietosc na przekroj(skonczony)

\textbf{dla $T_{1}$ mamy:}
\begin{enumerate}[label=(\alph*')]
\item Z opisu $T_{1}$ mamy, ze $ \mathbb{R} \in T_{1}$ oraz $\emptyset \in T_{1}$
\item Niech $U = \bigcup\limits_{\alpha \in A} U_{\alpha}$ gdzie $A_{\alpha} \in T_{1}$ oraz $A$ jest pewnym zbiorem indeksowym, bedzie dowolna suma(skonczona badz nie) zbiorow. Rozwazmy przypadki:
\begin{itemize}
\item $U_{\alpha} = \mathbb{R}$ dla pewnego $\alpha \in A$. 

Wtedy $ \bigcup\limits_{\alpha \in A} U_{\alpha} = \mathbb{R} \in T_{1}$
\item $U_{\alpha} \neq \mathbb{R}$ dla kazdego $\alpha \in A$.

Wtedy mamy przypadki:

\begin{itemize}
    \item $U_{\alpha} = \emptyset$ dla kazdego $\alpha \in A$. 

Wtedy $\bigcup\limits_{\alpha \in A} U_{\alpha} = \emptyset \in T_{1}$  
    \item $U_{\alpha} \neq \emptyset$ dla pewnego $\alpha \in A$ oraz zbior $$B = \left\{n: U_{\alpha} = (-n, n)  \land n \in \mathbb{N} \land \alpha \in A\right\}$$ jest ograniczony z gory.

Wtedy, z tego ze zbior $B$ jest ograniczony z gory i zawiera liczby naturalne, wiemy ze istnieje najmniejsza liczba naturalna $m$, ktora ogranicza ten zbior z gory. Wezmy zatem liczbe $m$. Zauwazmy, ze $m$ jest jednoczesnie maximum zbioru B. Z definicji zbioru $T_{1}$ mamy, ze $(-m, m) \in T_{1}$ oraz z tego, ze $B$ jest ograniczony z gory, wnioskujemy $$\bigcup\limits_{\alpha \in A} U_{\alpha} = (-m,m) \in T_{1}$$

    \item $U_{\alpha} \neq \emptyset$ dla pewnego $\alpha \in A$ oraz zbior $$B = \left\{n: U_{\alpha} = (-n, n) \land n \in \mathbb{N} \land \alpha \in A\right\}$$ jest nieograniczony z gory.

Wtedy, mam sprzecznosc, bo zalozylismy, ze $\mathbb{R} \notin T_{1}$, z drugiej strony zbior $B$ jest nieograniczony, a $$\bigcup\limits_{\alpha \in A}U_{\alpha} = \bigcup\limits_{n=1}^{\infty}(-n, n) = \mathbb{R}$$ 
Zatem ten podpunkt odpada.

\end{itemize}

Sprawdzilismy zatem wszystkie mozliwosci i otrzymalismy, ze zbior $T_{1}$ jest zamkniety na sume(skonczona badz nie)

\end{itemize}
\item Wezmy dwa dowolne zbiory $B \in T_{1}$ oraz $C \in T_{1}$. Pokazemy, ze $B \cap C \in T_{1}$

Rozwazmy przypadk:
\begin{itemize}
    
\item $B \lor C = \emptyset$

Wtedy: $B \cap C = \emptyset \in T_{1}$
\item $B \land C \neq \emptyset$ Wtedy mamy przypadki:

\begin{itemize}

\item $B \land C = \mathbb{R}$ 

Wtedy $B \cap C = \mathbb{R} \in T_{1}$

\item $B \lor C \neq \mathbb{R}$ 

Wtedy ktorys ze zbiorow(byc moze oba) jest postaci $(-m,m)$ dla $m \in \mathbb{N}$. Jesli $B = \mathbb{R}$ i $C = (-m,m), m\in \mathbb{N}$ to $B \cap C = (-m,m) \in T_{1}$.
Drugi przypadek to gdy oba sa postaci prostej, czyli $B = (-m,m)$ oraz $C = (-n,n)$ dla $m,n \in \mathbb{N}$. Wtedy niech $k = min(m,n)$, mamy $B \cap C = (-k,k) \in T_{1}$

\end{itemize}
\end{itemize}
Zatem $T_{1}$ zamkniete na przekroj.    
\end{enumerate}
$T_{1}$ jest zatem topologia. c.b.d.u.

\textbf{dla $T_{2}$ mamy:} 

Dowod bedzie analogiczny do tego z $T_{1}$. Postaram sie go bardziej ladnie zrobic.

\begin{enumerate}[label=(\alph*')]%poczatek dowodu dla T2

\item Z opisu $T_{2}$ wnioskujemy, ze $\mathbb{R} \in T_{2}$ oraz $\emptyset \in T_{2}$.
\item Niech $U = \bigcup\limits_{\alpha \in A}U_{\alpha}$, gdzie $U_{\alpha} \in T_{2}$ oraz $A$ jest pewnym zbiorem indeksowym. Rozwazmy przypadki:

\begin{itemize}%poczatek pierwszych przypadkow
   
\item $U_{\alpha} = \mathbb{R}$ dla pewnego $\alpha \in A$. Wtedy:

$U = \mathbb{R} \in T_{2}$
\item $U_{\alpha} \neq \mathbb{R}$ dla kazdego $\alpha \in A$ Wtedy:

Rozwazmy przypadki:
\begin{itemize}%poczatek drugich przypadkow

\item $U_{\alpha} = \emptyset$ dla kazdego $\alpha \in A$ Wtedy:

$U = \emptyset \in T_{2}$
\item $U_{\alpha} \neq \emptyset$ dla pewnego $\alpha$ Wtedy:

Wprowadzmy sobie funkcje zdaniowa $$P(x) = (\exists \alpha \in A)(U_{\alpha} = [-x,x])$$ oraz zbior $$B = \left\{n : P(n) \land n \in \mathbb{N}\right\}$$ jako zbior indeksow $n$ dla ktorych istnieje $\alpha \in A$ taka ze $U_{\alpha} = [-n,n]$.

Nastepnie rozwazmy dwa przypadki:

I. Zbior $B$ jest ogranioczony z gory.
Wtedy z tego, ze $B$ jest ograniczony oraz $B$ zawiera tylko liczby naturalne, wnioskujemy, ze istnieje w $B$ maximum. Niech $m = max(B)$. Wtedy $U = [-m,m] \in T_{2}$ 
\vspace{5mm}

II. Zbior $B$ jest nieograniczony z gory.
Wtedy $U = \bigcup\limits_{\alpha \in A}U_{\alpha} = \bigcup\limits_{n=1}^{\infty}[-n,n] = \mathbb{R}$, lecz zalozylismy, ze $U \neq \mathbb{R}$. Zatem sprzecznosc, wiec ten przypadek nam odpada.

\end{itemize}%koniec drugich przypadkow
Zatem $T_{2}$ jest zamkniety na sume.

\end{itemize}%koniec pierwszych przypadkow
\item Wezmy dowolne dwa zbiory $B \in T_{2}$ oraz $C \in T_{2}$. Wtedy:

Rozwazmy przypadki:

\begin{itemize}%poczatek trzecich przypadkow

\item $B = \mathbb{R} \land C = \mathbb{R}$. Wtedy:

$B \cap C = \mathbb{R} \in T_{2}$.
\item WLOG zalozmy, ze $B \neq \mathbb{R}$. Wtedy:

Rozwazmy przypadki:
\begin{itemize}%poczatek czwartych przypadkow
    
\item $B = \emptyset$. Wtedy bezwzgledu na $C$:

$B \cap C = \emptyset \in T_{2}$
\item $B = [-n,n]$ dla pewnego $n \in \mathbb{N}$. Wtedy:

I. Jesli $C = \emptyset$ to:

$B\cap C = \emptyset \in T_{2}$.
\vspace{5mm}

II. Jesli $C = [-m,m]$ dla pewnego $m \in \mathbb{N}$, Wtedy:

Niech $k = min(m, n)$, mamy $B\cap C = [-k,k] \in T_{2}$
\end{itemize}%koniec czwartych przypadkow
\vspace{5mm}

Zatem $T_{2}$ jest zamkniete na skonczone przekroje.  
\end{itemize}%koniec trzecich przypadkow

Podsumowujac, sprawdzilismy z definicji topologii i otrzymalismy ze $T_{2}$ jest topologia. c.b.d.u.

\end{enumerate}%koniec dowodu dla T2

\textbf{dla $T_{3}$ mamy:}

\begin{enumerate}[label=(\alph*')] %poczatek dowodu dla T3
    
\item Z opisu $T_{3}$ mamy, ze $\mathbb{R} \in T_{3}$ oraz $\emptyset \in T_{3}$
\item Niech $U = \bigcup\limits_{\alpha \in A}U_{\alpha}$, gdzie $U_{\alpha} \in T_{2}$ oraz $A$ jest pewnym zbiorem indeksowym.

Wprowadzmy sobie funkcje zdaniowa $$P(x) = (\exists \alpha \in A)(U_{\alpha} = [x, \infty])$$
oraz zbior
$$B = \left\{n: P(n) \land n \in {\mathbb{N}}^{+}\right\}$$
$B$ to zbior wszystkich indeksow $n$ dla ktorych $U_{\alpha} = [n, \infty]$.
Poniewaz zbior ten zawiera tylko liczby naturalne, wiec ma minimum. Niech $m = min(B)$. Wtedy $U = [m, \infty] \in T_{3}$. Zatem $T_{3}$ jest zamkniety na sume.
\item Wezmy sobie dowolne zbiory $B \in T_{3}$ oraz $C \in T_{3}$. Niech $b = min(B)$ oraz $c = min(C)$. Wtedy $B\cap C = [min(b,c), \infty] \in T_{3}$ bo $min(b,c) \in \mathbb{N}$. Zatem $T_{3}$ jest zamkniety na przekroj.

Podsumowujac, rozwazylismy wszystkie mozliwosci i otrzymalsmy, ze $T_{3}$ jest topologia.

\end{enumerate} %koniec dowodu dla T3

\item Niech $\mathbb{N}$ bedzie zbiorem liczb naturalnych dodatnich. Udowodnij ze kazda z nastepujacych kolekcji(w ramkach) podzbiorow $\mathbb{N}$ jest topologia.

\begin{tcolorbox}[colback=white!90!red,colframe=black!35!red,title=Definicja: \textbf{Initial segment topology}; poki co brak polskiego tlumaczenia]
$T_{1}$ zawiera $\mathbb{N}$ oraz $\emptyset$ oraz kazdy zbior $\left\{1,2,3,\dots,n\right\}$, dla dowolnego $n$ naturalnego dodatniego.
\end{tcolorbox}
\textbf{Dowod:}

\begin{itemize}%poczatek dowodu dla T1 
    
\item Z opisu mamy, ze $\emptyset \in T_{1}$ oraz $\mathbb{N} \in T_{1}$
\item Niech $U = \bigcup\limits_{\alpha \in A}U_{\alpha}$ dla pewnego zbioru indeksowego $A$ i $U_{\alpha} \in T_{1}$. Wprowadzmy sobie funkcje zdaniowa
$$P(x) = (\exists \alpha \in A)(U_{\alpha} = \left\{1,2,3,\dots,x\right\}$$
oraz zbior
$$B = \left\{n: P(n) \land n \in {\mathbb{N}}^{+}\right\}$$
Rozwazmy przypadki:
\begin{itemize}%poczatek przypadkow dla sumy
   
\item Zbior $B$ jest ogranioczony z gory.

Poniewaz $B$ zawiera liczby naturalne dodatnie i jest ograniczony z gory wiec
zawiera maximum. Niech $m = max(B)$. Zatem dla pewnego $\alpha, U_{\alpha} = \left\{1,2,\dots,m\right\}$. Wtedy $U = \left\{1,2,\dots,m\right\} \in T_{1}$.
\item Zbior $B$ jest nieograniczony z gory.

Poniewaz zbior $B$ zawiera liczby naturalne dodatnie(od 1 wzwyz) i nie jest ograniczony z gory, wiec $B = \mathbb{N}$. Zatem dla dowolnej $\alpha \in A$ istnieje ${\alpha}^{\prime} \in A$ taka ze, $U_{\alpha} \subset U_{{\alpha}^{\prime}}$. Zatem $U = \mathbb{N}$.
\end{itemize}%koniec przypadkow dla sumy
Zatem $T_{1}$ jest zamkniety na sume.
\item Wezmy dowolne dwa zbiory $B \in T_{1}$ oraz $C \in T_{1}$

Rozwazmy przypadki:
\begin{itemize}%poczatek przypadkow dla przekroju
    
\item $B = C = \mathbb{N}$. Wtedy:

$B\cap C = \mathbb{N} \in T_{1}$
\item WLOG $B \neq \mathbb{N}$. Wtedy:

Rozwazmy przypadki:
\begin{itemize}%poczatek drugich przypadkow dla przekroju

\item $B = \emptyset$ Wtedy:

Bezwzgledu na $C$. $B\cap C = \emptyset \in T_{1}$
\item $B \neq \emptyset$, czyli $B = \left\{1,2,\dots,n\right\}$ dla pewnego $n \in \mathbb{N}$ to:

Jesli $C = \emptyset$, to $B\cap C = \emptyset \in T_{1}$.

Jesli $C = \left\{1,2,\dots,m\right\}$ dla pewnego $m \in \mathbb{N}$, to $B\cap C = \left\{1,2,\dots,min(n,m)\right\} \in T_{1}$  

\end{itemize}%koniec drugich przypadkow dla przekroju
\end{itemize}%koniec przypadkow dla przekroju
Zatem $T_{1}$ jest zamkniety na przekroj.

Zatem $T_{1}$ to topologia.
\end{itemize}%koniec dowodu dla T1

\begin{tcolorbox}[colback=white!90!red,colframe=black!35!red,title=Definicja:\textbf{Final segment topology}; poki co brak polskiego tlumaczenia]
$T_{2}$ zawiera $\mathbb{N}$ oraz $\emptyset$ oraz kazdy zbior $\left\{n, n+1, n+2,\dots \right\}$, dla dowolnego $n$ naturalnego dodatniego.
\end{tcolorbox}
\textbf{Dowod:}

\begin{itemize}%poczatek dowodu dla T2
    
\item Z opisu mamy, ze $\emptyset \in T_{2}$ oraz $\mathbb{N} \in T_{2}$
\item Niech $U = \bigcup\limits_{\alpha \in A}U_{\alpha}$, gdzie $A$ jest pewnym zbiorem indeksowym oraz $U_{\alpha} \in T_{2}$. Wprowadzmy sobie funkcje zdaniowa
$$P(x) = (\exists \alpha \in A)(U_{\alpha} = \left\{x, x+1, x+2, \dots\right\}$$
oraz zbior
$$B = \left\{n: P(n) \land n \in {\mathbb{N}}^{+}\right\}$$

Niech $m = min(B)$. Wtedy dla kazdego $\alpha \in A, U_{\alpha} \subset \left\{m, m+1, \dots\right\}$ oraz $U = \left\{m, m+1, \dots\right\} \in T_{2}$.

Zatem $T_{2}$ jest zamkniety na sume.

\item Wezmy dowolne $B \in T_{2}$ oraz $C \in T_{2}$.

Rozwazmy przypadki:

\begin{itemize}%poczatek przypadkow dla przekroju
    
\item $B = \emptyset \lor C = \emptyset$. Wtedy:

$B\cap C = \emptyset \in T_{2}$
\item $B = \mathbb{N} \land C = \mathbb{N}$. Wtedy:

$B\cap C = \mathbb{N} \in T_{2}$

\item Zaden z nich nie jest pusty oraz oba nie sa zbiorem $\mathbb{N}$. Wtedy:

Jesli $B = \left\{m, m+1, \dots\right\}$ dla pewnego $m \in \mathbb{N}$ oraz $C =  \left\{n, n+1, \dots\right\}$ dla pewnego $n \in \mathbb{N}$, to $B\cap C = \left\{max(m,n), max(m,n)+1, \dots\right\} \in T_{2}$
\vspace{5mm}

Jesli WLOG $B = \left\{m, m+1, \dots\right\}$ dla pewnego $m \in \mathbb{N}$ oraz $C = \mathbb{N}$ to $B\cap C = \left\{m, m+1, \dots\right\} \in T_{2}$ 
\end{itemize}%koniec przypadkow dla przekroju
Zatem $T_{2}$ jest zamkniete na przekroj

Zatem $T_{2}$ to topologia. c.b.d.u.
\end{itemize}%koniec dowodu dla T2

\item Wypisz wszystkie mozliwe topologie na ponizszych zbiorach:

\begin{enumerate}[label=(\alph*)]%poczatek wymieniania zbiorow z polecenia
\item $X = \left\{a,b\right\}$

\textbf{Odpowiedz:}

Sa takie 4 topologie:
\begin{enumerate}[label={$T_{\arabic*} = $}]%poczatek wymieniania odpowiedzi
    
\item $\left\{X, \emptyset\right\}$
\item $\left\{X, \emptyset, \left\{a\right\}, \left\{b\right\} \right\}$
\item $\left\{X, \emptyset, \left\{b\right\} \right\}$
\item $\left\{X, \emptyset, \left\{a\right\} \right\}$
\end{enumerate}%koniec wymieniania odpowiedzi
 
\item $X = \left\{a,b,c\right\}$   

\textbf{Odpowiedz:}

Jest takich 29 topologii:TODO PRZEPISAC Z KARTKI JE TUTAJ
  
TODO NAPISZ DOWOD ZE JEST ICH 29, NA WIKI SPRAWDZ: TOPOLOGIES ON SET WITH 3 ELEMENTS
\begin{enumerate}[label = {$T_{\arabic*} = $}]%poczatek wymieniania odpowiedzi
    
\item $\left\{X, \emptyset\right\}$
\item $\left\{X, \emptyset, \left\{a\right\}, \left\{b\right\}, \left\{c\right\}, \left\{a,b\right\},\left\{a,c\right\},\left\{b,c\right\},\right\}$
\item $\left\{X, \emptyset, \left\{a\right\} \right\}$
\item $\left\{X, \emptyset, \left\{b\right\} \right\}$
\item $\left\{X, \emptyset, \left\{c\right\} \right\}$
\item $\left\{X, \emptyset, \left\{a\right\}, \left\{a,b\right\} \right\}$
\item $\left\{X, \emptyset, \left\{a\right\}, \left\{a,c\right\} \right\}$
\item $\left\{X, \emptyset, \left\{b\right\}, \left\{b,c\right\} \right\}$
\item $\left\{X, \emptyset, \left\{b\right\}, \left\{b,a\right\} \right\}$
\item $\left\{X, \emptyset, \left\{c\right\}, \left\{c,a\right\} \right\}$
\item $\left\{X, \emptyset, \left\{c\right\}, \left\{c,b\right\} \right\}$
\item $\left\{X, \emptyset, \left\{a\right\}, \left\{a,b,c\right\} \right\}$
\item $\left\{X, \emptyset, \left\{b\right\}, \left\{a,b,c\right\} \right\}$
\item $\left\{X, \emptyset, \left\{c\right\}, \left\{a,b,c\right\} \right\}$
\end{enumerate}%koniec wymieniania odpowiedzi 

\end{enumerate}%koniec wymieniania zbiorow z polecenia
\item Udowodnij ponizszy lemat.
\begin{tcolorbox}[colback=white!90!green,colframe=black!35!green,title=Lemat: Topologia dyskretna i podzbiory nieskonczone]

Niech $X$ bedzie zbiorem niesoknczonym oraz $T$ topologia na $X$. Jezeli kazdy nieskonczony podzbior $X$ nalezy do $T$, to $T$ jest topologia dyskretna.

\end{tcolorbox}
\textbf{Dowod:}

Zgodnie z ktoryms tam lematem(latwy dowod), jesli $T$ zawiera $\left\{x\right\}$ gdzie $x \in X$ jest dowolnym elementem $X$, to $T$ jest topologia dyskretna. 

Zatem wezmy dowolny element $x \in X$. Skoro $X$ jest nieskonczonym zbiorem i kazdy nieskonczony podzbior $X$ jest w $T$, to oznacza, ze mozemy sobie wybrac z $X$, nieskonczony ciag roznowartosciowy, w ktorym nie ma elementu $x$. 

Jak? Na przyklad tak:
$$
\begin{cases}
    a_{1} = \begin{cases} 1 \text{ jesli } x \neq 1 \\ 0 \text{ jesli } x = 1   \end{cases} \\
    a_{n} = min\left\{y: y \in X \land y \neq x \land (\forall i \in \left\{1,2,\dots,n-1\right\})(a_{i} \neq y)\right\}
\end{cases}$$

Nastepnie rozwazmy dwa nieskonczone zbiory:
$$A = \left\{a_{n}: n = 2\times k \land k \in \mathbb{Z}\right\} \cup \left\{x\right\}$$
$$\text{ oraz }$$ 
$$B = \left\{a_{n}: n = 2\times k + 1\land k \in \mathbb{Z}\right\} \cup \left\{x\right\} $$
Wtedy oczywiscie $A \in T$ oraz $B \in T$ oraz $A \cap B = \left\{x\right\} \in T$

Z dowolnosci $x$, wnioskujemy, ze $T$ zawiera dowolny singleton $\left\{x\right\}$ taki ze, $x \in X$. Zatem $T$ jest topologia dyskretna. c.b.d.u. 

\item Wskaz i udowodnij, ktore z ponizszych zbior liczb rzeczywistych $\mathbb{R}$ sa topologiami.

\begin{enumerate}[label={$T_{\arabic*} :$}]
\item zawiera $\mathbb{R}, \emptyset$ oraz kazdy przedzial $(a,b)$ gdzie $a,b \in \mathbb{R}$ oraz $a < b$;
\item zawiera $\mathbb{R}, \emptyset$ oraz kazdy przedzial $(-r,r)$ gdzie $r \in {\mathbb{R}}^{+}$;
\item zawiera $\mathbb{R}, \emptyset$ oraz kazdy przedzial $(-r,r)$ gdzie $r \in {\mathbb{Q}}^{+}$;
\item zawiera $\mathbb{R}, \emptyset$ oraz kazdy przedzial $[-r,r]$ gdzie $r \in {\mathbb{Q}}^{+}$;
\item zawiera $\mathbb{R}, \emptyset$ oraz kazdy przedzial $(-r,r)$ gdzie $r \in {\mathbb{R}}^{+} \setminus \mathbb{Q}$;
\item zawiera $\mathbb{R}, \emptyset$ oraz kazdy przedzial $[-r,r]$ gdzie $r \in {\mathbb{R}}^{+} \setminus \mathbb{Q}$;
\item zawiera $\mathbb{R}, \emptyset$ oraz kazdy przedzial $[-r,r)$ gdzie $r \in {\mathbb{R}}^{+}$;
\item zawiera $\mathbb{R}, \emptyset$ oraz kazdy przedzial $(-r,r]$ gdzie $r \in {\mathbb{R}}^{+}$;
\item zawiera $\mathbb{R}, \emptyset$, kazdy przedzial $[-r,r]$ oraz kazdy przedzial $(-r,r)$ gdzie $r \in {\mathbb{R}}^{+}$;
\item zawiera $\mathbb{R}, \emptyset$, kazdy przedzial $[-n,n]$ oraz kazdy przedzial $(-r,r)$ gdzie $n \in {\mathbb{N}}^{+}$ oraz  $r \in {\mathbb{R}}^{+}$;
\end{enumerate}


\end{enumerate}%koniec numeracji zadan

\begin{tcolorbox}[colback=white!90!red,colframe=black!35!red,title=1.2.1 Definicja: Zbior otwarty- open set]

Niech $(X, T)$ bedzie przestrzenia topologiczna. Wtedy elementy $T$ nazywaja sie \textbf{zbiory otwarte.}

\end{tcolorbox}

\begin{tcolorbox}[colback=white!90!green,colframe=black!35!green,title=1.2.2 Lemat: Przestrzen topologiczna i zbiory otwarte]
Niech $(X, T)$ bedzie przestrzenia topologiczna. Wtedy

\begin{itemize}%poczatek lematu
       
\item $X$ oraz $\emptyset$ sa zbiorami otwartymi.
\item suma(skonczona lub nie) zbiorow otwartych jest zbiorem otwartym
\item przekroj skonczony zbiorow otwartych jest zbiorem otwartym
\end{itemize}%koniec lematu
\end{tcolorbox}
\textbf{Dowod:}
Wynika to wprost z definicji. Podpunkt trzeci wynika z zadania 4.

\begin{tcolorbox}[colback=white!90!red,colframe=black!35!red,title=1.2.3 Definicja: Zbior zamkniety- closed set]

Niech $(X,T)$ bedzie przestrzenia topologiczna. Podzbior $S$ zbioru $X$ jest \textbf{zbiorem zamknietym} w $(X, T)$ jesli jego dopelnienie w $X$, $X\setminus S$, jest zbiorem otwartym w $(X, T)$
\end{tcolorbox}
\textbf{Komentarz}

Czyli mowimy, ze $S$, bedace podzbiorem $X$, jest zbiorem zamknietym w $(X,T)$ jesli jego dopelnienie w $X$, czyli $X\setminus S$ jest otwarte, czli jesli $X\setminus S \in T$

Zauwazmy, ze jesli $(X,T)$ jest przestrzenia dyskretna, wtedy kazdy podzbior $X$ jest zbiorem zamknietym. Jednakze w przestrzeni niedyskretnej, $(X,T)$, jedynymi zbiorami zamknietymi sa $X$ oraz $\emptyset$

\begin{tcolorbox}[colback=white!90!green,colframe=black!35!green,title=1.2.5 Lemat: Przestrzen topologiczna i zbiory zamkniete]

Niech $(X,T)$ bedzie przestrzenia topologiczna. Wtedy

\begin{itemize}%poczatek lematu
\item $\emptyset$ oraz $X$ sa zbiorami zamknietymi
\item przekroj skonczonej lub nieskonczonej liczby zbiorow zamknietych jest zbiorem zamknietym
\item suma skonczonej liczby zbiorow zamknietych jest zbiorem zamknietym
    
\end{itemize}%koniec letmatu
\end{tcolorbox}
\textbf{Komentarz:}

Mozna tu dostrzec pewna analogie pomiedzy przestrzenia topologiczna i zbiorami otwartymi

\textbf{Dowod:}

TODO

\end{document} %end calego dokumentu
