\documentclass{article}

\usepackage{import}
\usepackage{transparent}
\usepackage{xcolor}
\usepackage{tcolorbox}
\usepackage{amsmath}
\usepackage{upgreek}
\usepackage{enumitem}
\usepackage{amssymb}
\usepackage[margin=0.25in]{geometry}

%Commands definitions
\newcommand{\setbackgroundcolour}{\pagecolor[rgb]{0.19, 0.19, 0.19}}
\newcommand{\settextcolour}{\color[rgb]{0.87, 0.87, 0.87}}
\newcommand{\invertbackgroundtext}{\setbackgroundcolour\settextcolour}

%Command execution
%If this line is commented, then the appearance remaind as usual.
\invertbackgroundtext

\author{Lukasz Kopyto}
\title{Topologia, notatki z wykladu.}

\begin{document}
\maketitle

\section{Wyklad nr 1. 20.02.2024}


\subsection{Metryka i przestrzen metryczna.}
Metryka, czyli uogolnienie odleglosci. Chcemy mierzyc odleglosc miedzy elemenetami zbioru $X$, obojetnie czym on by nie byl. O elementach tego zbioru myslimy jak o punktach.
\begin{tcolorbox}[colback=white!90!red,colframe=black!35!red,title=Definicja 1.1: Metryka]

    Funkcja $d: X \times X \rightarrow [0; \infty)$ jest metryka, jezeli
         \begin{itemize}
             \item $d(x,y) = 0 \iff x = y$
             \item $(\forall x,y) (d(x,y) = d(y,x))$
             \item $(\forall x,y,z) (d(x,z) \leq d(x,y) + d(y,z))$ 
         \end{itemize}
\end{tcolorbox}

\textbf{Komentarz:}
Pierwsze dwa punkty definicji 1.1 sa oczywiste, chcemy aby odleglosc miedzy tymi samymi punktami byla zerowa oraz odleglosc miedzy x oraz y byla taka sama jak miedzy y oraz x. Rdzeniem definicji metryki, jest punkt 3 jej definicji, nierownosc trojkata. Mianowicie, jesli podczas naszej drogi, miedzy punktami x oraz z, chcemy odwiedzic dodatkowo punkt y, to naturalne jest, ze nasza droga moze sie wydluzyc.  

\textbf{Przyklady metryk:}
TODO


\begin{tcolorbox}[colback=white!90!red,colframe=black!35!red,title=Definicja 1.2: Przestrzen metryczna]
    Niech $X$ bedzie pewnym zbiorem oraz funkcja $d$ bedaca metryka okreslona na tym zbiorze. Wtedy para $(X, d)$ nazywamy przestrzen metryczna. 
\end{tcolorbox}
\textbf{Komentarz:}
Intuicyjnie o powyzszej parze mozemy myslec jak o zbiorze wyposazonym w pewna strukture, podobnie zbior i czesciowy porzadek tworza zbior czesciowo uporzadkowany(poset- partially ordered set)

\begin{tcolorbox}[colback=white!90!red,colframe=black!35!red,title=Definicja 1.3: Kula]

    Niech $(X,d)$ bedzie przestrzenia metryczna. Kula o promineiu $r>0$ i srodku $x\in X$ nazywamy zbior $$B_{r}(x) = \left\{ y \in X: d(x,y) < r \right\} $$

\end{tcolorbox}

\textbf{Przyklady kul w roznych metrykach:}
TODO

\begin{tcolorbox}[colback=white!90!red,colframe=black!35!red,title=Definicja 1.4: Zbieznosc ciagu.]

    Niech $(X,d)$ bedzie przestrzenia metryczna. Ciag elementow tej przestrzeni $(x_{n})$ jest zbiezny do $x$, jezeli:
    $$(\forall \epsilon > 0)(\exists N)(\forall n > N) (d(x_{n}, x) < \epsilon) $$
\end{tcolorbox}

\begin{tcolorbox}[colback=white!90!red,colframe=black!35!red,title=Definicja 1.5 Zbieznosc ciagu. Alternatywna definicja]

    Niech $(X,d)$ bedzie przestrzenia metryczna. Ciag elementow tej przestrzeni $(x_{n})$ jest zbiezny, jesli \textbf{kazda kula o srodku w x zawiera prawie wszystkie wyrazy $(x_{n})$}

\end{tcolorbox}
\textbf{Przyklady:}
TODO

\begin{tcolorbox}[colback=white!90!red,colframe=black!35!red,title=Definicja 1.6 Zbior otwarty.]

    Niech $(X,d)$ bedzie przestrzenia metryczna. Zbior $U \subset X$ jest otwarty, jezeli:
    $$(\forall x \in U)(\exists r > 0) (B_{r}(x) \subseteq U)$$
\end{tcolorbox}

\begin{tcolorbox}[colback=white!90!green,colframe=black!35!green,title=Fakt 1.7: Zbior otwarty jest suma kul]

    Zbior $U$ jest otwarty $\iff U$ jest suma kul.
    (Intuicja: zbior otwarty czyli przedzialy i ich sumy.)
\end{tcolorbox}
\textbf{Dowod:} TODO

\textbf{Komentarz:}

Zbior jest otwarty jezeli dla dowolnego wyboru punktu z tego zbioru moge wybrac kule o srodku w tym punkcie. 

Zbior pusty i cala przestrzen jest zawsze otwarta. 

W przestrzeni dyskretnej kazdy zbior jest otwarty.

\begin{tcolorbox}[colback=white!90!green,colframe=black!35!green,title=Fakt 1.8: Sumy i przekroje zbiorow otwartych.]
\begin{itemize}
    \item Suma dowolnie wielu zbiorow otwartych jest otwarta,
    \item Przekroj skonczenie wielu zbiorow otwartych jest otwarty.
\end{itemize}
\end{tcolorbox}
\textbf{Dowod:} TODO

\begin{tcolorbox}[colback=white!90!red,colframe=black!35!red,title=Definicja 1.9: Zbior domkniety.]

    Niech $(X,d)$ bedzie przestrzenia metryczna. Zbior $F$ jest domkniety, jezeli kazdy zbiezny ciag elementow $F$ ma granice w $F$.
\end{tcolorbox}

\begin{tcolorbox}[colback=white!90!green,colframe=black!35!green,title=Fakt 1.10: Zbior domkniety kiedy]
    
    Zbior $U$ jest otwarty $\iff$ $U^{c}$ jest domkniety.
\end{tcolorbox}

\textbf{Dowod:} TODO
\section{Wyklad nr 2. 27.02.2024}

TODO pojecia z wykladu do zrobienia:

\begin{itemize}

\item Wnetrze, Brzeg, Domkniecie
\item Definicja ciaglosci w punkcie x
\item Ciaglosc w sensie cauchy'eqo
\item NWSR: f ciagla <-> dla kazdego x jesli $x_{n}$ zbiega do x do $f(x_{n})$ zbiega d0 $f(x)$ w sensie Heinego <-> przeciwobraz zbioru U jest owtarty dla kazdego U otwartego w Y
\item homeomorfizm
\item topologia
\item
    \begin{itemize}
        upgrade z przestrzeni metrycznej do topologicznej $(X,T)$
    \item $F \subseteq X$ domkniety jesli $F^{c} \in T$
    \item $x_n \rightarrow x$ jezeli $\forall x \in U \exists N \forall n > N x_{n} \in U$
    \item $IntA$ jest najwiekszy w sensie zawierania zbior otwarty w $A$
    \item $A$ z kreska na gorze najmniejszy domkniety zbior zawarty w $A$ 
    \item f ciagla $f^{-1}[U]$ otwarty $\forall U otwarty$
    \end{itemize}
\item prosta sorgenfreya
\item prosta hausdorffa

\end{itemize}



\end{document}
